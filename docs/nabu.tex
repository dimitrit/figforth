\documentclass{article}
\usepackage[letterpaper, total={7.5in, 9in}]{geometry}
\usepackage{multicol}
\usepackage{enumitem}
\title{NABU Extensions}
\author{D Theulings}
\date{September 2023}
\newcommand{\n}{\textit{n}}
\setlist[description]{%
	align=left,
	leftmargin=2.5em,
	labelwidth=2em,
	font={\bfseries\ttfamily},
}
\begin{document}
	\maketitle
	\begin{multicols}{2}[]
		\setlength{\parskip}{.5em}
		\setlength\parindent{0pt}

		\section{Introduction}
		Support for NABU video and audio features\footnotemark{} is included
		in \verb|NABU.FTH|. After loading the screens, use \verb|NABU| to invoke
		the appropriate vocabulary. Refer to the NABU-HOWTO document for details 
		on configuring and using fig-FORTH on the NABU PC.

		\section{Glossary}
		\begin{description}
			\item[\#MOTION]\texttt{n --- }\\
				Sets sprite numbers 0 to \textit{n} - 1 in auto motion.

			\item[BEEP]\texttt{--- }\\
				Produces the sound associated with correct input or prompting.

			\item[CHAR]\texttt{n1 n2 n3 n4 char --- }\\
				Defines character no. \textit{char} to have the pattern specified by
				the four numbers \verb|(n1, n2, n3, n4)| on the stack. The definition
				for character \textit{\#0} by default resides at 800h. Each character
				definition is 8 bytes long with each number on the stack representing
				two bytes.

			\item[CHARPAT]\texttt{char --- n1 n2 n3 n4 }\\
				Places the four number (8-byte) pattern of a specified character char
				on the stack. By default, the definition for character \textit{\#0}
				resides at 800h.

			\item[COINC]\texttt{spr1 spr2 tol --- flag }\\
				Detects a coincidence between two given sprites within a specified
				tolerance limit \textit{tol}. A true flag indicates a coincidence.

			\item[COINCALL]\texttt{--- flag }\\
				Detects a coincidence between the visible portions of any two sprites
				on the display screen. A true flag indicates a coincidence.

			\item[COINCXY]\texttt{dotcol dotrow spr tol --- flag }\\
				Detects a coincidence between a specified sprite and a given point
				\verb|(dotcol,dotrow)| within a given tolerance limit \textit{tol}.
				A true flag indicates a coincidence.

			\item[COLOR]\texttt{n1 n2 n3 --- }\\
				Causes a specified character set \textit{n3} to have the given
				foreground \textit{n1} and background \textit{n2} colors.

			\item[COLTAB]\texttt{--- vaddr }\\
				A constant whose value is the beginning VDP address of the color
				table. The default value is 380h.

			\item[CURPOS]\texttt{--- addr }\\
				A user variable that stores the current VDP cursor position.

			\item[DCOLOR]\texttt{--- addr }\\
				A variable which contains the dot-color information used by \verb|DOT|.
				Its value may be a two-digit hexadecimal number which defines the
				foreground and background color, or it may be -1 which means no
				color information is changed in the VDP.

			\item[DDOT]\texttt{dotcol dotrow --- b vaddr }\\
				The assembly code routine called by \verb|DOT|. It expects a dot
				column and a dot row on the stack and returns a byte \textit{b} with
				only one bit set and a VDP address \textit{vaddr}. The dot referenced
				by \verb|(dotcol,dotrow)| is translated by \verb|DDOT| to the address
				\textit{vaddr} of the byte containing it and a mask \textit{b} that 
				locates the dot within the byte.

			\item[DELALL]\texttt{--- }\\
				Delete all sprites.

			\item[DELSPR]\texttt{spr --- }\\
				Delete the specified sprite \textit{spr}.

			\item[DMODE]\texttt{--- addr }\\
				A variable that determines which dot mode is currently in effect.
				A \verb|DMODE| value of 0 indicates \textit{DRAW} mode, a value of 1
				indicates \textit{UNDRAW} mode, and a value of 2 indicates
				\textit{DOT-TOGGLE} mode. This variable is set by the \verb|DRAW|,
				\verb|UNDRAW| and \verb|DTOG| words.

			\item[DOT]\texttt{dotcol dotrow --- }\\
				Plots a dot at \verb|(dotcol,dotrow)| in whatever mode is selected
				by \verb|DMODE| and in whatever color is selected by \verb|DCOLOR|.

			\item[DRAW]\texttt{--- }\\
				Sets \verb|DMODE| equal to 0. This means that dots are plotted in
				the `on' state.

			\item[DTOG]\texttt{--- }\\
				Sets \verb|DMODE| equal to 2. This means that each dot plotted
				takes on the opposite state of the dot currently at that location.

			\item[DXY]\texttt{dotcol1 dotrow1 dotcol2 dotrow2 --- n1 n2 }\\
				Places on the stack the square of the x distance \textit{n1} and
				the square of the y distance \textit{n2} between the points
				\verb|(dotcol1,dotrow1)| and \verb|(dotcol2,dotrow2)|.

			\item[GCHAR]\texttt{col row --- char }\\
				Returns on the stack the ASCII code \textit{char} of the character
				currently at \verb|(col,row)|. Note that rows and columns are
				numbered from 0.

			\item[GOTOXY]\texttt{col row --- }\\
				Places the cursor at the designated column \textit{col} and row
				position \textit{row}. Note that rows and columns are numbered
				from 0.

			\item[GRAPHICS]\texttt{--- }\\
				Converts from present display screen mode into standard Graphics mode
				configurations.

			\item[GRAPHICS2]\texttt{--- }\\
				Converts from present Forth screen mode into standard Graphics2 mode
				configuration.

			\item[HCHAR]\texttt{col row count char --- }\\
				Prints a horizontal stream of a specified character \textit{char}
				beginning at \verb|(col,row)| and having a length \textit{count}.
				Note that rows and columns are numbered from 0.

			\item[HONK]\texttt{--- }\\
				Produces the sound associated with incorrect input.

			\item[JOYST]\texttt{n1 --- char n2 n3 }\\
				Allows you to accept input from joystick \#1 and the left side of
				the keyboard (\textit{n1 = 1}) or from joystick \#2 and the right
				side of the keyboard (\textit{n1 = 2}). Values returned are
				the character code \textit{char} of the key pressed, the x value
				\textit{n2} and the y value \textit{n3}.

			\item[LINE]\texttt{dotcol1 dotrow1 dotcol2 dotrow2 --- }\\
				The high resolution graphics routine which plots a line from
				\verb|(dotcol1,dotrow1)| to \verb|(dotcol2,dotrow2)|. \verb|DCOLOR|
				and \verb|DMODE| must be set before this instruction is used.

			\item[MAGNIFY]\texttt{n --- }\\
				Alters the sprite magnification factor to be \textit{n}. The value
				of \textit{n} must be 0, 1, 2 or 3.

			\item[MCHAR]\texttt{n col row --- }\\
				Places a square of color \textit{n} at \verb|(col,row)|. Used in
				multicolor mode.

			\item[MINIT]\texttt{--- }\\
				Initializes the monitor screen for use with \verb|MCHAR|.

			\item[MOTION]\texttt{n1 n2 spr --- }\\
				Assigns a horizontal \textit{n1} and vertical \textit{n2} velocity
				to the specified sprite \textit{spr}.

			\item[MULTI]\texttt{--- }\\
				Converts from present display screen mode into standard Multicolor
				mode configuration.

			\item[PDT]\texttt{--- vaddr }\\
				A constant which contains the VDP address of the Pattern Descriptor
				Table. Defaul tvalue is 800h.

			\item[SATR]\texttt{--- vaddr }\\
				A constant whose value vaddr is the VDP address of the Sprite
				Attribute List. Default value is 300h.

			\item[SCREEN]\texttt{n --- }\\
				Changes the display screen color to the color specified \textit{n}.
				The foreground (FG) and background (BG) screen colors must be placed
				in the low-order byte of \textit{n}, with FG the high-order 4 bits
				and BG the low-order 4 bits, e.g., \textit{n = 27} (1Bh) for black
				on light yellow.

			\item[SMTN]\texttt{--- vaddr }\\
				A constant whose value is the VDP address of the Sprite Motion
				Table. Default value is 780h.

			\item[SPCHAR]\texttt{n1 n2 n3 n4 char --- }\\
				Defines a character \textit{char} in the Sprite Descriptor Table to
				have the pattern composed of the 4 words (cells) on the stack.

			\item[SPDTAB]\texttt{--- vaddr }\\
				A constant whose value is the VDP address of the Sprite Descriptor
				Table. Default value is 800h. Note that this coincides with the
				Pattern Descriptor Table.

			\item[SPLIT]\texttt{--- }\\
				Converts from present display screen mode into standard Split mode
				configuration.

			\item[SPLIT2]\texttt{--- }\\
				Converts from present display screen mode into standard Split2 mode
				configuration.

			\item[SPRCOL]\texttt{n spr --- }\\
				Changes color of the given sprite number \textit{spr} to the color
				\textit{n} specified.

			\item[SPRDIST]\texttt{spr1 spr2 --- n }\\
				Returns on the stack the square of the distance \textit{n} between
				two specified sprites, \textit{spr1} and \textit{spr2}. Distance is
				measured in pixels and the maximum distance that can be detected
				accurately is 181 pixels.

			\item[SPRDISTXY]\texttt{dotcol dotrow spr --- n }\\
				Places on the stack \textit{n}, the square of the distance between
				the point \verb|(dotcol,dotrow)| and a given sprite \textit{spr}.
				Distance is measured in pixels and the maximum distance that
				can be detected accurately is 181 pixels.

			\item[SPRGET]\texttt{spr --- dotcol dotrow }\\
				Returns the dot column \textit{dotcol} and dot row \textit{dotrow}
				position of sprite \textit{spr}.

			\item[SPRITE]\texttt{dotcol dotrow n char spr --- }\\
				Defines sprite number \textit{spr} to have the specified location
				\verb|(dotcol,dotrow)|, color \textit{n}, and character pattern
				\textit{char}. The size of the sprite will depend on the magnification
				factor.

			\item[SPRPAT]\texttt{char spr --- }\\
				Changes the character pattern of a given sprite \textit{spr} to
				\textit{char}.

			\item[SPRPUT]\texttt{dotcol dotrow spr --- }\\
				Places a given sprite \textit{spr} at location \verb|(dotcol,dotrow)|.



			\item[SSDT]\texttt{vaddr --- }\\
				Places the Sprite Descriptor Table at the specified VDP address
				\textit{vaddr} and initializes all sprite tables. The address given
				must be on an even 2K boundary. This instruction must be executed
				before sprites can be used.

			\item[SYS\$]\texttt{--- addr }\\
				A user variable that contains the address of the system support entry point.

			\item[SYSTEM]\texttt{n --- }\\
				Calls the system synonyms. You must specify an offset \textit{n} into
				a jump table for the routine you wish to call. The offset \textit{n}
				must be one of the predefined even numbers. See system Forth screen 
				\textbf{TODO} for offsets \textbf{...}.

			\item[VAND]\texttt{b vaddr --- }\\
				Performs a logical AND on the byte at the specified VDP location
				\textit{vaddr} and the given byte \textit{b}. The result byte is
				stored back into the VDP address.

			\item[VCHAR]\texttt{col row count char --- }\\
				Prints on the display screen a vertical stream of length \textit{count}
				of the specified character \textit{char}. The first character of the
				stream is located at \verb|(col,row)|. Rows and columns are numbered
				from 0 beginning at the upper left of the display screen.

			\item[VFILL]\texttt{vaddr count b --- }\\
				Fills \textit{count} locations beginning at the given VDP address
				\textit{vaddr} with the specified byte \textit{b}.

			\item[VMBR]\texttt{vaddr addr count --- }\\
				Reads \textit{count} bytes beginning at the given VDP address
				\textit{vaddr} and places them at \textit{addr}.

			\item[VMBW]\texttt{addr vaddr count --- }\\
				Writes \textit{count} bytes from \textit{addr} into VDP memory
				beginning at the given VDP address \textit{vaddr}.

			\item[VOR]\texttt{b vaddr --- }\\
				Performs a logical OR on the byte at the specified VDP address
				\verb|vaddr| and the given byte \textit{b}. The result byte is stored
				back into the VDP address.

			\item[VSBR]\texttt{vaddr --- b }\\
				Reads a single byte from the given VDP address \textit{vaddr} and places
				its value \textit{b} on the stack.

			\item[VSBW]\texttt{b vaddr --- }\\
				Writes a single byte \textit{b} into the given VDP address \textit{vaddr}.

			\item[VWTR]\texttt{b n --- }\\
				Writes the given byte \textit{b} into the specified VDP write-only register
				\textit{n}.

			\item[VXOR]\texttt{b vaddr --- }\\
				Performs a logical XOR on the byte at the specified VDP address
				\textit{vaddr} and the given byte \textit{b}. The result byte is
				stored back into the VDP address \textit{vaddr}.
		\end{description}

		\footnotetext{NABU Network, \textit{NABU Personal Computer Technical Specifications}, (Ottawa, ON: NABU Manufacturing Corporation, 1984), ⟨URL:\\ https://cloud.nabu.ca/docs/Nabu\_PC\_Technical\_Specifications.pdf⟩ – visited on 2023-08-29.}
	\end{multicols}
\end{document}