\documentclass{article}
\usepackage[letterpaper, total={7.5in, 9in}]{geometry}
\usepackage{multicol}
\usepackage{enumitem}
\title{RomWBW Extensions}
\author{D Theulings}
\date{August 2023}
\newcommand{\n}{\textit{n}}
\setlist[description]{%
	align=left,
	leftmargin=2.5em,
	labelwidth=2em,
	font={\bfseries\ttfamily},
}
\begin{document}
\maketitle
\begin{multicols}{2}[]
	\setlength{\parskip}{.5em}
	\setlength\parindent{0pt}

	\section{Introduction}
	Support for RomWBW HBIOS features\footnotemark{} is included in \verb|ROMWBW.FTH|.
	Use \verb|ROMWBW| to invoke the	appropriate vocabulary.

	\section{Glossary}
	\begin{description}
		\item[.B]\texttt{n --}\\
			Print BCD value \n{}, converted to decimal. No following blank is printed.

		\item[AT]\texttt{col row ---}\\
			Position the text cursor at the given position. Both column and
			row positions are zero indexed, thus \verb|0 0 AT| will move the
			cursor to the top left of the screen. Note that \verb|AT| does
			\textbf{not} update \verb|OUT|.

		\item[CLS]\texttt{---}\\
			Clear VDU screen.

		\item[KEY?]\texttt{--- c t ¦ f}\\
			Check if a key has been pressed. Returns false if no key has been
			pressed. Returns true and the key's ascii code if a key has been
			pressed.

		\item[STIME]\texttt{addr ---}\\
			Set the RTC time. \textit{addr} is the address of the 6 byte
			date/time buffer \verb|YMDHMS|. Each byte in this buffer is BCD
			encoded.

		\item[TIME]\texttt{--- addr}\\
			Get the RTC time and leave the address of the 6 byte date/time
			buffer \verb|YMDHMS|. Each byte in this buffer is BCD encoded.
	\end{description}

	\footnotetext{Wayne Warthen, \textit{RomWBW Architecture}, (RetroBrew Computers Group, 2020), ⟨URL:\\ https://www.retrobrewcomputers.org/lib/exe/fetch.php?media=software:firmwareos:romwbw:romwbw\_architecture.pdf⟩ – visited on 2023-08-29.}
\end{multicols}
\end{document}