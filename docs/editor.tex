\documentclass{article}
\usepackage[letterpaper, total={6in, 8in}]{geometry}
\usepackage{multicol}
\usepackage{enumitem}
\title{EDITOR USER MANUAL}
\author{Bill Stoddart\\of FIG, United Kingdom}
\date{} % Nov 1984?

\newcommand{\n}{\textit{n}}
\setlist[description]{%
	align=left,
	leftmargin=2.5em,
	labelwidth=2em,
	font={\bfseries\ttfamily},
}
\begin{document}
\maketitle
%\tableofcontents
\begin{multicols}{2}[]
	\setlength{\parskip}{.5em}
	\setlength\parindent{0pt}
	\section{Introduction}
	FORTH organizes its mass storage into ``screens'' of 1024 characters.
	If, for example, a diskette of 250k byte capacity is used entirely
	for storing text, it will appear to the user as 250 screens
	numbered 0 to 249.

	Each screen is organized as 16 lines with 64 characters per line.
	The FORTH screens are merely an arrangement of virtual memory and
	need not correspond exactly with the screen format of a particular
	terminal.

	\section{Selecting a Screen and Input of Text}

	To start an editing session the user types \verb|EDITOR| to invoke the
	appropriate vocabulary.

	The screen to be edited is then selected, using either:
	\begin{description}
	\item[n LIST]
		list screen \n{} and select it for editing, \textit{or}
	\item[n CLEAR]
		clear screen \n{} and select for editing.
	\end{description}

	To input new text to screen \n{} after \verb|LIST| or \verb|CLEAR|
	the \verb|P| (put) command is used.

	Example:
	\begin{verbatim}
	O P THIS IS HOW
	1 P TO INPUT TEXT
	2 P TO LINES O, 1, AND 2 OF THE SCREEN.
	\end{verbatim}

	\section{Line Editing}

	During this description of the editor, reference is made to \verb|PAD|.
	This is a text buffer which may hold a line of text used by or
	saved with a line editing command, or a text string to be found or
	deleted by a string editing command.

	\verb|PAD| can be used to transfer a line from one screen to another,
	as well as to perform edit operations within a single screen.

	\subsection{Line Editor Commands}
	\begin{description}
		\item[n H]
			Hold line \n{} at \verb|PAD|
			(used by system more often than by user).
		\item[n D]
			Delete line \n{} but hold it in \verb|PAD|.
			Line 15 becomes blank as lines \n{}+1 to 15 move up 1 line.
		\item[n T]
			Type line \n{} and save it in \verb|PAD|.
		\item[n R]
			Replace line \n{} with the text in \verb|PAD|.
		\item[n I]
			Insert the text from \verb|PAD| at line \n{},
			moving the old line \n{} and following lines down.
			Line 15 is lost.
		\item[n E]
			Erase line \n{} with blanks.
		\item[n S]
			Spread at line \n{}. \n{} and subsequent lines
			move down 1 line. Line \n{} becomes blank.
			Line 15 is lost.
	\end{description}

	\section{Cursor Control and String Editing}

	The screen of text being edited resides in a buffer area of
	storage. The editing cursor is a variable holding an offset into
	this buffer area. Commands are provided for the user to position
	the cursor, either directly or by searching for a string of buffer
	text, and to insert or delete text at the cursor position.

	\subsection{Commands to Position the Cursor}
	\begin{description}
		\item[TOP]
			Position the cursor at the start of the screen.
		\item[n M]
			Move the cursor by a signed amount \n{} and
			print the cursor line. The position of the cursor on
			its line is shown by a \verb|_| (underline).
	\end{description}

	\subsection{String Editing Commands}
	\begin{description}
		\item[F text]
			Search forward from the current cursor position
			until string `\textit{text}' is found. The cursor is left at
			the end of the text string, and the cursor line is printed.
			If the string is not found an error message is given
			and the cursor is repositioned at the top of screen.
		\item[B]
			Used after \verb|F| to back up the cursor by the length
			of the most recent text.
		\item[N]
			Find the next occurrence of the string found by an
			\verb|F| command.
		\item[X text]
			Find and delete the string `\textit{text}'.
		\item[C text]
			Copy in text to the cursor line at the cursor position
		\item[TILL text]
			Delete on the cursor line from the cursor till the
			end of the text string `\textit{text}'.
	\end{description}
	\textbf{NOTE:} Typing \verb|C| with no text will copy a \verb|NULL| character
	into the text at the cursor position. This will abruptly stop subsequent
	compilation! To delete this error type \verb|TOP X| `return'.

	\subsection{Screen Editing Commands}
	\begin{description}
		\item[n LIST]
			List screen \n{} and select it for editing
		\item[n CLEAR]
			Clear screen \n{} with blanks and select it for editing
		\item[n1 n2 COPY]
			Copy screen \textit{n1} to screen \textit{n2}.
		\item[L]
			List the current screen. The cursor line is relisted
			after the screen listing, to show the cursor position.
		\item[FLUSH]
			Used at the end of an editing session to ensure that
			all entries and updates of text have been transferred
			to disk.
	\end{description}
\end{multicols}
\textbf{}
\end{document}